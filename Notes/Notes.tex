
\documentclass[12pt]{article}
\input{MyPreamble}
\usepackage{multirow}
\usepackage{amsmath}
\usepackage{fancyhdr}


\usepackage[left=1in,top=1.0in,right=1in,headsep=0.25in,footskip=0.3in,bottom=0.7in]{geometry}
%\usepackage[left=1in,top=1.2in,right=1in,footskip=0.3in,bottom=0.7in,showframe]{geometry}
%\usepackage[left=1in,top=1in,right=1in,bottom=1in,nohead]{geometry}
\usepackage{graphicx}
\usepackage{gensymb}
\renewcommand{\arraystretch}{1} % spacing between table rows
\usepackage[]{caption}
\setlength{\abovecaptionskip}{0pt}
\setlength{\belowcaptionskip}{-5pt}
\setlength{\intextsep}{10pt plus 2pt minus 2pt}
\usepackage[normalem]{ulem}
\newcounter{Labcounter}
\newcounter{Taskcounter}
\numberwithin{equation}{section}
\fancyhf{}
\pagestyle{fancy}
\fancypagestyle{plain}{ %
% \fancyhf{} % remove everything
\renewcommand{\headrulewidth}{0pt} % remove lines as well
%\renewcommand{\footrulewidth}{0pt}
%\cfoot{Page \thepage~of \pageref{LastPage}}}
\cfoot{\thepage}
}
\usepackage[numbered,framed]{matlab-prettifier}
\lstMakeShortInline[style=Matlab-editor]"

\def\code#1{\texttt{#1}}

\newcommand{\blue}[1]{\textcolor{blue}{#1}} %for displaying red texts
%\newcommand{\rood}[1]{} %for displaying red texts
\newcommand{\red}[1]{\textcolor{red}{[#1]}} %for displaying red texts


\lhead{\textit{Nanospheres...}}
%\chead{\Large \textbf{Paul W.~Leu } \vspace{0.3em}}
%\chead{\Large \textbf{NSF Biographical Sketch: Paul W.~Leu} \vspace{0.3em}}
\rhead{\textit{Leu}}


\newcommand{\vectornorm}[1]{\left|\left|#1\right|\right|}
%\usepackage[top=2.5cm, bottom=2.5cm, left=2.5cm, right=2.5cm]{geometry}
\usepackage[normalem]{ulem}
\newenvironment{packed_enum}{
\begin{enumerate}
  \setlength{\topsep}{0pt}
  \setlength{\partopsep}{0pt}
  \setlength{\itemsep}{1pt}
  \setlength{\parskip}{0pt}
  \setlength{\parsep}{0pt}
}{\end{enumerate}}

%\usepackage[small]{caption}
\usepackage[draft]{pdfcomment}
\usepackage{wrapfig}
\usepackage{hyperref}
\usepackage{paralist}
\usepackage{amsmath}
\usepackage{amssymb}
\usepackage{amsfonts}
\usepackage{textcomp}
%\usepackage{subfig}
\usepackage{subcaption}
\usepackage{framed}
\usepackage{setspace}
\usepackage{here}
\usepackage[numbers, square, comma, sort&compress]{natbib}

\usepackage[compact]{titlesec}
\titlespacing{\section}{0pt}{0ex}{0pt}
\titlespacing{\subsection}{0pt}{0pt}{0pt}
\usepackage{xcolor}

\usepackage[]{caption}
\setlength{\abovecaptionskip}{0pt}
\setlength{\belowcaptionskip}{-5pt}
\setlength{\intextsep}{10pt plus 2pt minus 2pt}


\usepackage{float}
\floatstyle{plaintop}
\newfloat{program}{thp}{lop}
\floatname{program}{Table}

\newfloat{wrapprogram}{thp}{lop}
\floatname{wrapprogram}{Table}

%\setlength{\intextsep}{10pt plus 2pt minus 2pt}
% bold face: highlight keywords, or big ideas
% italics: inconspicuous stressing of key points
% underline: hypothesis; avoid

\usepackage{bm}

\date{\today }

\author{
Paul W. Leu\\
University of Pittsburgh\\
Pittsburgh, PA}


%\def\myTitle{CAREER: Transforming Solar Energy Harvesting through Nanophotonic Light Trapping}
%\def\myTitle{CAREER: Characterizing Enhanced Absorption and Carrier Collection Mechanisms in Silicon and Zinc Oxide Nanocone-based Solar Cells}
%Upper bound
%   100 hours/year
%    40 hours
%    
%\title{Ultimate Limits of Silicon Nanostructures for Photon Management\\
%\title{Determining Silicon/Metal Nanostructures that Approach the Wave-Optics Light Trapping Limit by Data Mining of Electrodynamic Simulations}
%\title{CAREER: Characterizing Enhanced Absorption and Carrier Collection Mechanisms in Silicon and Zinc Oxide Nanocone-based Solar Cells}
%\title{Predicting Wave-Optics Light Trapping in Silicon and Metal Nanostructure by Data Mining of Electrodynamic Simulations}
%Keywords: ultra thin, light trapping, wave-optics light trapping, coherent light trapping, optimization framework, data mining}


\graphicspath{{../Paper/}{../Figures/}{Figures/}}

%\title{Modeling and Manufacturing of Three Dimensional Silicon Nanostructures for Photon Management}

\begin{document}

\tableofcontents

\section{Simulations}

We are using PVSyst for simulations.  
The information on IAM loss can be found here \url{https://www.pvsyst.com/help/iam_loss.htm}.  

For normal incidence, the reflection is of the order of 5\% and is included in the measured STC performance.  
The standard test condition performance is 1000 W/m$^2$ and is 1000 W/m$^2$, 25 $\degree$C, and AM 1.5.  

%The full year simulation may be studied in detail using the ``detailed study" button in the ``detailed losses" dialog.  

For normal glass, the Fresnel model uses 

Ashrae parameterization uses the following formula
\begin{equation}
F_{IAM}  =  1  -  b_o \left (\frac{1}{\cos \theta_i} - 1 \right )
\end{equation}
where $b_0 = 0.1$.  

Typical IAM loss is between 3 to 4\%.  
% https://aurorasolar.com/blog/understanding-pv-system-losses-part-4-tilt-orientation-incident-angle-modifier-environmental-conditions-and-inverter-losses-clipping/

According to \url{https://pvpmc.sandia.gov/modeling-steps/1-weather-design-inputs/shading-soiling-and-reflection-losses/incident-angle-reflection-losses/}, 
the $IAM$ or angle of incidence modifier for beam component of incident irradiance is defined as  
\begin{align}
IAM_B &= \frac{T(\theta)}{T(0)} \\
&= \frac{1 - R(\theta) - A(\theta)}{1 - R(0) - A(0)} \\
&= \frac{1 - R(\theta)}{1 - R(0)} 
\end{align}


\begin{enumerate}
\item Do the simulations with $IAM_B = \frac{1 - R(\theta)}{1 - R(0)}$.  
\item Renormalize all numbers upward by factor so that normal glass is so with $IAM = 0$ at normal incidence.  
\item Check enhancements.  
\end{enumerate}

Karinna is using
\begin{equation}
IAM_B = 1 - R(\theta)
\end{equation}
Get data.  We can divide by $1 - R_{glass} (0)$.  
% about 3.5 % reflection for nanocone at 40 degrees latitude
% about 13% reflection for bare glass at 40 degrees latitude

\bibliographystyle{IEEEtran}
\bibliography{AllRefs}


\end{document}
I